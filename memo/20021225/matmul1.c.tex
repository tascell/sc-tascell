\documentstyle{jarticle}
\begin{document} \begin{verbatim}
#include<stdio.h>
#include<string.h>
#include<stdlib.h>
#include<ctype.h>


typedef int     (mat[2][2]);

void 
matmul(mat a, mat b, mat c)
{
  int             i;
  int             j;
  int             k;
  unsigned char   k0;
  int             i0 = 0;
  const int       i0_init = 0;
  int             j0 = 0;
  const int       j0_init = 0;
  int             i1 = 0;
  const int       i1_init = 0;
  int             j1 = 0;
  const int       j1_init = 0;
  int             k1 = 0;
  const int       k1_init = 0;
  for ((i0 = i0_init); i0 <= 1; i0++) {
    for ((j0 = j0_init); j0 <= 1; j0++) {
      c[i0][j0] = 0;
    }
  }
  for ((i1 = i1_init); i1 <= 1; i1++) {
    for ((j1 = j1_init); j1 <= 1; j1++) {
      for ((k1 = k1_init); k1 <= 1; k1++) {
        typedef int     l;
        c[i1][j1] += a[i1][k1] * b[k1][j1];
      }
    }
  }
  return;
}

int 
main()
{
  mat             a = {{1, 2}, {3, 4}};
  const mat       a_init = {{1, 2}, {3, 4}};
  mat             b = {{5, 6}, {7, 8}};
  const mat       b_init = {{5, 6}, {7, 8}};
  mat             c;
  {
    a[0][0] = a_init[0][0], a[0][1] = a_init[0][1], a[1][0] = a_init[1][0], a[1][1] = a_init[1][1];
    b[0][0] = b_init[0][0], b[0][1] = b_init[0][1], b[1][0] = b_init[1][0], b[1][1] = b_init[1][1];
    matmul(a, b, c);
    printf("a:%d %d %d %d \n", a[0][0], a[0][1], a[1][0], a[1][1]);
    printf("b:%d %d %d %d \n", b[0][0], b[0][1], b[1][0], b[1][1]);
    printf("c:%d %d %d %d \n", c[0][0], c[0][1], c[1][0], c[1][1]);
    return;
  }
}
\end{verbatim} \end{document}
